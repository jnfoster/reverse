\documentclass[11pt]{article}
% decent example of doing mathematics and proofs in LaTeX.
% An Incredible degree of information can be found at
% http://en.wikibooks.org/wiki/LaTeX/Mathematics

% Use wide margins, but not quite so wide as fullpage.sty
\marginparwidth0.5in 
\oddsidemargin0.25in 
\evensidemargin0.25in 
\marginparsep0.25in
\topmargin 0.25in 
\textwidth 6in \textheight 8 in
% That's about enough definitions

\usepackage{amsmath}
\usepackage{upgreek}
\usepackage{enumerate}
\usepackage{graphicx}
\usepackage{enumitem}
\usepackage{amssymb}
\usepackage{mathtools}
\usepackage{array}
\newcolumntype{P}[1]{>{\centering\arraybackslash}p{#1}}
\usepackage{multirow,bigdelim}

\begin{document}
\title{Theorems to Prove}
\author{CS 4999: Anna Yesypenko Supervised by Professor Nate Foster}
\maketitle

We introduce the following notation: $$\vec{x} \triangleq (\hat{\rho}_x \in \texttt{program}, \hat{\sigma}_x \in \texttt{stack}, \rho_x \in \texttt{program}, \sigma_x \in \texttt{stack})$$

Based on the linear contraints of our stack machine, we have the following property about well-formed closures:
\begin{align*}
    \texttt{wf}\ (\vec{x}, v \in \texttt{stack\_value}) \quad \triangleq \quad (\texttt{Apply}::\rho, v::\texttt{Closure}\ (\hat{\rho}, \hat{\sigma})::\sigma, \dots) &\Longrightarrow\\
    (\hat{\rho}\ @\ \rho, v::\hat{\sigma}\ @\ \sigma, \dots) &\Longrightarrow^* (\rho, v'::\sigma, \dots)
\end{align*}
Recall that to reverse $\texttt{Apply}$, we must split the program and the stack such that we restore the original closure. We define this split by the following:
\begin{align*}
    \texttt{splits}\ (\rho', \sigma', v, \vec{x}) \quad \triangleq \quad \rho' = \hat{\rho}_x\ @\ \rho_x \wedge \sigma' = \hat{\sigma}_x\ @\ \sigma_x: \ \texttt{wf}\ (\vec{x}, v)
\end{align*}
Using the property of the well-formed closure, if we could prove the $\texttt{Unique Splits}$ proposition, then $\texttt{Apply}$ would be injective.
\begin{align*}
    \texttt{Unique Splits}: \quad \forall \rho', \sigma'.\ \exists\ \vec{x}\ \texttt{splits}\ (\rho', \sigma', \vec{x}) \wedge \exists\ \vec{y}\ \texttt{splits}\ (\rho', \sigma', \vec{y}) \Longrightarrow \vec{x} = \vec{y}
\end{align*}



\end{document}
