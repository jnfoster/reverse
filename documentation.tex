\documentclass[11pt]{article}

\marginparwidth0.5in 
\oddsidemargin0.25in 
\evensidemargin0.25in 
\marginparsep0.25in
\topmargin 0.25in 
\textwidth 6in \textheight 8 in

\usepackage{amsmath}
\usepackage{upgreek}
\usepackage{enumerate}
\usepackage{verbatim}
\usepackage{stmaryrd}
\usepackage{listings}

\newcommand*{\br}[1]{\llbracket{#1}\rrbracket}


\begin{document}
\title{Linear Lambda Calculus Compiler Documentation}
\author{CS 4999: Anna Yesypenko Supervised by Professor Nate Foster}
\maketitle

\section*{Overview:}

We compile the source language, linear lambda calculus programs $e$
\begin{align*}
    b\ &::=\ +\ |\ -\ |\ /\ |\ *\\
    e\ &::=\ n\ |\ x\ |\ e_1\ b\ e_2\ |\ \text{let}\ x\ =\ e_1\ \text{in}\ e_2\\
    &|\ \lambda\ x.\ e\ |\ e_1\ e_2
\end{align*}
to the target language, stack machine programs $\rho$
\begin{align*}
    binop\ &::=\ \text{Add $|$ Subt $|$ Mult $|$ Div}\\
    funop\ &::= \text{Save\_Function ($fun\_id, k$) $|$ Form\_Closure ($fun\_id$) $|$ Apply}\\
    rollop\ &::= \text{Roll $k$ $|$ Unroll $k$}\\
    tupleop\ &::= \text{Construct\_Tuple $k$ $|$ Deconstruct\_Tuple $k$}\\
    instr\ &::=\ binop\ |\ funop\ |\ rollop\ |\ tupleop\ |\ \text{Push $n$}\\
    \rho &::=\ instr\ \text{list}
\end{align*}
which operate on stacks $\sigma$ and $\sigma_f$
\begin{align*}
    \sigma_f\ &::=\ (fun\_id, \rho)\ \text{list}\\
    tup\ &::=\ \text{Tuple $(v_1, \dots, v_k)$}\\
    v\ &::=\ \text {Int $n\ |\ tup\ |$ Closure ($\rho, tup$)}\\
    \sigma &::=\ v\ \text{list}
\end{align*}

\break{}

\section*{Defunctionalization:}

Given a lambda-calculus program $p$, defunctionalization produces a first-order language. That is, functions are no longer considered to be values. Instead, $l\ ::=\ (\lambda x.\ e)$ is represented as $C_l(v_1, \dots, v_n)$, where $C_l$ uniquely identifies the function $l$, and $v_1, \dots, v_n$ are the values of the variables $x_1, \dots, x_n$ which are free in $l$.

We define the translation from lambda-calculus program to first-order program.
\begin{align*}
        \br{x} &= x\\
            \br{\lambda x.\ e} &= C_{\lambda x.\ e} (x_1, \dots, x_n),\ \text{where $x_1, \dots, x_n$ are free in $\lambda x.\ e$}\\
                \br{e_1\ b\ e_2} &= \br{e_1}\ b\ \br{e_2}\\
                    \br{\text{let}\ x = e_1\ \text{in}\ e_2} &= \br{(\lambda x.\ e_2)\ e_1}\\
                        \br{e_1\ e_2} &= \text{apply}(\br{e_1}, \br{e_2})
\end{align*}

Thus a lambda calculus program $p$ is translated to a first-order program as such:
\begin{lstlisting}[mathescape = true]
let rec apply_defunc (f, arg) = 
    match f with
        | C$_{\lambda x.\ e} (x_1, \dots, x_n)$ -> let $x = \arg$ in $\br{e}$
        | $\dots$ in
        $\br{p}$
\end{lstlisting}

\section*{Analogue to Defunctionalization:}

The values Closure ($fun\_id$, Tuple $(v_1, \dots, v_k)$) on stack $\sigma$ are analogous to the constructors $C_{fun\_id} (v_1, \dots, v_k)$.\\

The function stack $\sigma_f$ stores the programs $\hat{\rho}$ corresponding to each case of the dispatch function `apply\_defunc'. At the beginning of a program $\rho$, there will be a series of Save\_Function instructions to initialize $\sigma_f$.

\break{}

\section*{Executing and Reversing a Program:}

The rules we present below show the reversibility for each instruction using the following tuple: $$(\rho, \sigma, \sigma_f, \sigma_h \in \text{int list}, \rho_h).$$

\textbf{Binomial arithmetic operations:}
\begin{gather*}
    (\text{(Add $|$ Subt $|$ Mult $|$ Div as $b$)}::\rho, \text{Int $n_1$}::\text{Int $n_2$}::\sigma, \sigma_f, \sigma_h, b::\rho_h)\\
    \Longleftrightarrow (\rho, n_1 \star n_2::\sigma, \sigma_f, n_1::\sigma_h, b::\rho_h)
\end{gather*}

\textbf{Function operations:}
\begin{gather*}
    (\text{Save\_Function ($fun\_id, k$) as $sf$}::\hat{\rho}\ @\ \rho, \sigma, \sigma_f, \sigma_h, \rho_h), \text{where $|\hat{\rho}| = k$}\\
    \Longleftrightarrow (\rho, \sigma, (fun\_id, \hat{\rho})::\sigma_f, \sigma_h, sf::\rho_h)\\ \\
    (\text{Form\_Closure ($fun\_id$) as $fc$}::\rho, \text{Tuple $(v_1, \dots, v_k)$ as $tup$}::\sigma, \sigma_f, \sigma_h, \rho_h)\\ \text{where List.assoc $fun\_id\ \sigma_f = \hat{\rho}$}\\
    \Longleftrightarrow (\rho, \text{Closure ($\hat{\rho}, tup$)}::\sigma, \sigma_f, \sigma_h, fc::\rho_h)\\ \\ 
    (\text{Apply}::\rho, \text{Closure ($\hat{\rho}$, Tuple $(v_1, \dots, v_k)$ as $tup$)}::\arg::\sigma, \sigma_f, \sigma_h, \rho_h),\\     
    \Longleftrightarrow (\hat{\rho}\ @\ \rho, tup::\arg::\sigma, \sigma_f, |\hat{\rho}|::\sigma_h, fc::\rho_h)
\end{gather*}


\textbf{Roll, Tuple, and Integer Pushing operations:}
\begin{gather*}
    (\text{Roll $k$}::\rho, v_1::v_2::\dots::v_k::\sigma, \sigma_f, \sigma_h, \rho_h)\\
    \Longleftrightarrow (\rho, v_k::v_1::\dots::v_{k-1}::\sigma, \sigma_f, \sigma_h, \text{Roll $k$}::\rho_h)\\ \\
    (\text{Unroll $k$}::\rho, v_1::v_2::\dots::v_k::\sigma, \sigma_f, \sigma_h, \rho_h)\\
    \Longleftrightarrow (\rho, v_2::\dots::v_k::v_1::\sigma, \sigma_f, \sigma_h, \text{Unroll $k$}::\rho_h)\\ \\
    (\text{Compose\_Tuple $k$}::\rho, v_1:: \dots:: v_k::\sigma, \sigma_f, \sigma_h, \rho_h)\\
    \Longleftrightarrow (\rho, \text{Tuple}(v_1, \dots, v_k)::\sigma, \sigma_f, \sigma_h, \text{Compose\_Tuple $k$}::\rho_h)\\ \\
    (\text{Decompose\_Tuple $k$}::\rho, \text{Tuple}(v_1, \dots, v_k)::\sigma, \sigma_f, \sigma_h, \rho_h)\\
    \Longleftrightarrow (\rho, v_1::\dots::v_k::\sigma, \sigma_f, \sigma_h, \text{Decompose\_Tuple $k$}::\rho_h)\\ \\
    (\text{Push $n$}::\rho, \sigma, \sigma_f, \sigma_h, \rho_h)\\
    \Longleftrightarrow (\rho, \text{Int $n$}::\sigma, \sigma_f, \sigma_h, \text{Push $n$}::\rho_h)
\end{gather*}

\section*{Reversability of Apply:}

All the operations change the size of the stack $\sigma$.
\begin{align*}
        (\text{Add}\ |\ \text{Subt}\ |\ \text{Mult}\ |\ \text{Div})\ &\rightarrow\ -1\\
        \text {Save\_Function} &\rightarrow\ +0\\
        \text {Form\_Closure} &\rightarrow\ +0\\
        \text {Apply}\ &\rightarrow\ -1\\
        \text{Roll}\  |\ \text{Unroll} \ &\rightarrow\ +0\\
        \text {Compose\_Tuple}\ k\ &\rightarrow\ -k + 1\\
        \text {Decompose\_Tuple}\ k\ &\rightarrow\ +k - 1\\
        \text {Push}\ &\rightarrow\ +1\\
\end{align*}

Therefore, we can classify the instructions into grow ops and shrink ops.
\begin{align*}
    i_g ::=& \text{Push | Compose\_Tuple 0 | Decompose\_Tuple $k > 1$}\\
    \rho_g ::=& \text{$i_g$ list}\\
    i_s ::=&\ \text {Add}\ |\ \text {Subt}\ |\ \text {Mult}\ |\ \text {Div}\\
    & |\ \text{Save\_Function \_ | Form\_Closure \_ | Apply}\\
    & |\ \text{Roll}\ n\ |\ \text {Unroll}\ n\}\\
    & |\ \text{Compose\_Tuple $k>0$ | Decompose\_Tuple $k \le 1$ | Push \_}\\
    \rho_s ::=&\ i_s\ \text{list}
\end{align*}

Let us define the following property.
\begin{align*}
        \textbf{Grow-Shrink Property}:\\
            \text{gs}\ (\rho\ \text{where}\ |\rho| > 0)\ = \ \exists \rho_g,\ \rho_s.\ \rho = \rho_g\ @\ \rho_s
\end{align*}
    
        The translation we produce has important properties that may allow Apply to be injective. For every program $\rho$ produced by the translation, gs ($\rho$) holds. Additionally, for every closure Closure ($n, \hat{\rho})$, gs ($\hat{\rho}$) holds. 
\end{document}
